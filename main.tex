% !TEX program = latexmk
% !TEX encoding = UTF-8 Unicode

% 注意:请使用 TeXLive 2019 及以上版本,已在 2019 2020 2021 版本测试编译成功,已在 2018 测试编译失败
% 如不想更新,可尝试使用 手动编译 XeLaTeX(可能出错)
% WARNING: Please use TeXLive version >= 2019, This template has been tested and found available on 2019, 2020, 2021, unavailable on 2018. 

% 注意:请确保自己已经完整阅读了 README.md 这一 markdown 文档,部分功能的使用方法并未在 .tex 中直接给出;如果仍有使用问题,请在 Github 上提出 issue
% WARNING: Pleans read README.md first cause some usages are not given in .tex files.

%%%%%%%%%%%%%%%%%%%%%%%%%%%%%%%%%%%%%%%%%%%%%%%%%%%%%%%%%%%%%%%%%%%%%%%%%%%%%%%%%%%%%%%%%%%%%%%%%%%%%
%% select the basic style of this thesis/dissertation (this document will use thesis for convenience)
%% 选择论文的基本类型

\documentclass[
    doctor,     % 必选项:   {master, doctor} 此处不区分专业/学术学位,在下面学科类型处区分
                % Mandatory: {master, doctor} No difference between Academic degree and Professional degree,
                %                             but be careful about the arguments in `\degree' and `\subject'
    % english,  % 可选项:   英文正文请选择此项
                % Optional:  For english main content, It will change some auto-generated matter into English
    % blind,    % 可选项:   论文用于盲审请选择此项,将不会产生致谢(内容)、答辩委员会会议决议(内容)、
                %            常规评阅人名单(内容),注意请自行修改 `攻读学位期间取得的研究成果' 内的内容格式
                % Optional:  For blind review, It will not generate Acknowledgements(Content), Decision of Defense Committee(Content),
                %            General Reviewers List(Content), CHANGE THE STYLE of Achievements BY YOURSELF
]{XJTU-thesis}



%%%%%%%%%%%%%%%%%%%%%%%%%%%%%%%%%%%%%%%%%%%%%%%%%
%% fill the each blank for auto-generate contents
%% 填写以下信息用以自动生成

% 论文标题(不超过35个字,英文注意大小写规律)
% Title, Make sure you have an acceptable capitalization
\title{这是一个不超过三十五个字的名字比较长的关于如何开展XXX的研究}{English Title and English Title and English Title}

% 学位类型(请按照 'Materials/Requirements/2021/01 中英文题名页示例/英文标准翻译/' 路径下文档正确填写
% Type of your degree, Translate it from documents in `Materials/Requirements/2021/01 中英文题名页示例/英文标准翻译/'
\degree{航空宇航科学与技术}{Aeronautical and Astronautical Science and Technology}

% 作者姓名(注意:所有人名英文均为「名在前,姓在后」,如果只有外文名,请在两个参数都填写外文名称)
% If you have only foreign name, put it as both first and second argument
\author{郑正正}{Zhengzheng Zheng}

% 指导老师姓名(注意:同作者姓名)
% Name of supervisor, It have the same requirments as the author
\advisor{冯峰}{教授}{Feng Feng}{Prof.}

% 合作指导老师姓名 或 老师团队(合作指导老师指:1.与招生简章中一致的合作导师,2.CSC项目的合作导师)
%(校方模板要求只能选择一个,都有则显示合作导师)
% Name of associate advisor or adviror's team, You can use only one of them, and advisorassociate has a higher priority.
% \advisorassociate{陈尘}{副教授}{Chen Chen}{Asscociate Prof.}
% \advisorteam{团队中文名}{English Name of the Team}

% 学科名称
% Name of the subject, also get it from that file
\subject{航空宇航科学与技术}{Aeronautical and Astronautical Science and Technology}

% 答辩委员会委员 显示的顺序和这里的一样 第一个人是主席
% Committee member of your defence, notice that the order shows in the thesis is same as here, and the first one is the chairperson
% each member should be put as {Organization,Name,Title} split by comma
\addcommitteemember{西安交通大学,张长长,教授}
\addcommitteemember{西安理工大学,王旺旺,教授}
\addcommitteemember{国网陕西经济技术研究院,李力,高工}
\addcommitteemember{西安交通大学,东方不败,副教授}
\addcommitteemember{西安交通大学,赵照,研究员}

% 答辩时间(手动指定)
% Defence date, input manually
\defensedate{2021}{06}{22}

% 答辩地点(涉密论文请手动设置为「西安交通大学」)
% Defence location, default value is 「西安交通大学」
\defenseloc{西安交通大学主楼E座303室}

% 论文提交日期,不输入参数则默认使用当前日期,如手动指明年月,请在第一个可选参数内填写年份,第二个可选参数填写月份(均为阿拉伯数字)
% Submission date of this thesis, if you not put it manually, it will use the current time
% \submitdate[2021][06]
\submitdate

% 常规审阅人 要求和答辩委员会委员一样
% General reviewer list, same requirements as the committee member
\addgeneralreviewer{西安交通大学,张长长,教授}
\addgeneralreviewer{西安理工大学,王旺旺,教授}
\addgeneralreviewer{国网陕西经济技术研究院,李力,高工}
\addgeneralreviewer{西安交通大学,东方不败,副教授}
\addgeneralreviewer{西安交通大学,赵照,研究员}

% 学院
% School or Faculty, unused now
% \school{电气工程学院}{School of Electrical Engineering}

% 专业[学士学位使用]
% Major, unused now
% \stumajor{计算机科学与计数}{Computer Science}

% 学号[学士学位使用]
% Student ID, unused now
% \stuid{}

% 班级[学士学位使用]
% Administrative class, unused now
% \adminclass{电气7xx班}{}

\addbibresource{References/reference.bib}

%%%%%%%%%%%%%%%%%%%%%%%%%%%%%%%%%%%%%%%%%%%%
%% 如果有使用其他包,请在这里添加
%% If you need other packages, use them here

% \usepackage{}


%%%%%%%%%%%%%%%%%%%%%%%%%%%%%%%%%%%%%%%%%%%%%%%%%%%%%%%%%%%%%%%%%%%%%%%%%%%%%%%%%%%
%% 注意:根据校方要求,以下所有页面顺序不可调整
%% Notice: The order of these pages are defined in the requirements of the University.

\begin{document}
% [自动生成] 中英题名页
% [Auto Generate] Chinese English Title Page
\thesistitles

% [自动生成] 答辩委员会页
% [Auto Generate] Defense Committee Pages
\thesiscommittes

% 生成摘要页 修改 Main_Miscellaneous/abstract_chs/eng.tex 中的内容
% Abstract, Rewrite your content in Main_Miscellaneous/abstract_chs/eng.tex
\thesisabstract

% 中英目录页
% Table of Contents
\thesistableofcontens

% 主要符号表 修改 Main_Miscellaneous/glossary.tex 中的内容
% Glossaries Page, Rewrite your content in Main_Miscellaneous/glossary.tex
\thesisglossarylist

% 正文 注意:英文正文写作时,中、英标题还是先中后英标题;同时,下面参数的顺序有意义,不要乱放
% Main contents, X of cX is the chapter of the thesis, you can change it if you want, the order matters
% and KEEP Chinese Title as the FIRST argument, English Titile as the SECOND, if you use `english' option

\thesisbody{
    Main_Spine/c1,
    Main_Spine/c2,
    Main_Spine/c3,
    Main_Spine/c4,
    % Main_Spine/c5
}

% 致谢 修改 Main_Miscellaneous/acknowledegment.tex 中的内容
% Acknowledgement, Rewrite your content in Main_Miscellaneous/acknowledegment.tex
\thesisacknowledegment

% TODO 修改
% [自动生成] 参考文献 默认使用 References/reference.bib,手动指定请使用可选参数 [path/to/bibfile]
% [Auto Generate] Bibliography, Default file is References/reference.bib, use argument [path/to/bibfile] for manual specification
\thesisbibliography

% 附录(有几个附录就导入几个文件(不加.tex后缀)),
% Appendi(x/ies), argument should not have the .tex suffix
\thesisappendix{Main_Miscellaneous/appendix_a,
                Main_Miscellaneous/appendix_b}

% 攻读学位期间取得的研究成果 修改 Main_Miscellaneous/achievement.tex 中的内容 注意盲审时需要手动修改格式
% Achievement, Rewrite your content in Main_Miscellaneous/achievement.tex
\thesisachivements

% 答辩委员会决议 修改 Main_Miscellaneous/decision.tex 中的内容
% Decision, Rewrite your content in Main_Miscellaneous/decision.tex
\thesisdecision

% [自动生成] 常规评阅人名单 需要手动指定两个数字
% [Auto Generate] Reviewer List, set two number as the content in this page 
\thesisreviewers{7}{5}

% [自动生成] 独创性声明
% [Auto Generate] Originality Declaration
\thesisdeclarations

\end{document}