% !TEX program = latexmk
% !TEX encoding = UTF-8 Unicode

% 注意:请使用 TeXLive 2020 及以上版本,已在 2020 2021 版本测试编译成功,已在 2019 测试编译失败
% 如不想更新,可尝试使用 手动编译 XeLaTeX(可能出错)
% WARNING: Please use TeXLive version >= 2020, This template has been tested and found available on 2020, 2021, unavailable on 2019. 

% 注意:请确保自己已经完整阅读了 README.md 这一 markdown 文档,部分功能的使用方法并未在 .tex 中直接给出;如果仍有使用问题,请在 Github 上提出 issue
% WARNING: Pleans read README.md first cause some usages are not given in .tex files.

%%%%%%%%%%%%%%%%%%%%%%%%%%%%%%%%%%%%%%%%%%%%%%%%%%%%%%%%%%%%%%%%%%%%%%%%%%%%%%%%%%%%%%%%%%%%%%%%%%%%%
%% select the basic style of this thesis/dissertation (this document will use thesis for convenience)
%% 选择论文的基本类型

\documentclass[
    doctor,     % 必选项:   {master, doctor} 此处不区分专业/学术学位,在下面学位类型处区分
                % Mandatory: {master, doctor} No difference between Academic degree and Professional degree,
                %                             but be careful about arguments in `\degree' and `\subject'
    % english,  % 可选项:   英文正文请选择此项
                % Optional:  For english main content, It will change some auto-generated matter into English
    % blind,    % 可选项:   论文用于盲审请选择此项,将隐去致谢页内容(但保留标题)、答辩委员会会议决议页内容(同前)、
                %            常规评阅人名单页内容(同前),将隐去题名页、答辩委员会页所有人名,
                %            注意:如果非自动读取成果数据库,请自行修改 `攻读学位期间取得的研究成果' 内的内容格式
                % Optional:  For blind review, It will not generate Acknowledgements(Content), Decision of Defense Committee(Content),
                %            General Reviewers List(Content), hide all names. NOTICE: CHANGE THE STYLE of Achievements BY YOURSELF
    plgck,      % 可选项:   论文用于查重请选择此项,将只产生从摘要到附录(含)的内容,且所有图片均不显示
                % Optional:  For plagiarism check, It will only produce content from abstract to appendicies, 
                %            meanwhile all figures will not be displayed.
]{XJTU-thesis}



%%%%%%%%%%%%%%%%%%%%%%%%%%%%%%%%%%%%%%%%%%%%%%%%%
%% fill the each blank for auto-generate contents
%% 填写以下信息用以自动生成

% 论文标题(不超过35个字,英文注意大小写规律)
% Title, Make sure you have an acceptable capitalization
\title{这是一个不超过三十五个字的名字比较长的关于如何开展XXX的研究}{English Title and English Title and English Title}

% 学位类型
% 考虑到专业学位的学位名称的少数特例(如专业型法学硕士不是 Master of Juris 而是 Juris Master),此处学位类型请按照案例和文件填写
% 同时为了前向兼容,可选项「A/P」代表「学术/专业」型学位,默认为学术型
% Type of your degree, Translate it from documents in `Materials/Requirements/2021/01 中英文题名页示例/英文标准翻译/'
% \degree[A]{硕士}{Master of Engineering} % 学术型(Academic)硕士请基于 '学术学位名称.txt' 填写
% \degree[P]{工程硕士}{Master of Engineering} % 专业型(Professional)硕士请基于 '专业学位(领域)英文标准翻译.pdf' 填写
%\degree[A]{博士}{Doctor of Philosophy}  % 学术型(Academic Doc)博士请填写 '{博士}{Philosophy}'
\degree[P]{工程博士}{Doctor of Engineering} % 专业型(Professional Doc)博士请基于 '专业学位(领域)英文标准翻译.pdf' 填写

% 作者姓名(注意:所有人名英文均为「名在前,姓在后」,如果只有外文名,请在两个参数都填写外文名称)
% If you have only foreign name, put it as both first and second argument
\author{郑正正}{Zhengzheng Zheng}

% 指导老师姓名(注意:同作者姓名)
% Name of supervisor, It have the same requirments as the author
\advisor{冯峰}{教授}{Feng Feng}{Prof.}

% 合作指导老师姓名 或 老师团队(合作指导老师指:1.与招生简章中一致的合作导师,2.CSC项目的合作导师)
%(校方模板要求只能选择一个,都有则显示合作导师)
% Name of associate advisor or adviror's team, You can use only one of them, and advisorassociate has a higher priority.
% \advisorassociate{陈尘}{副教授}{Chen Chen}{Asscociate Prof.}
% \advisorteam{团队中文名}{English Name of the Team}

% 学科名称,请基于 '学科(专业)英文标准翻译.pdf' 填写
% Name of the subject, also get it from that file
% \subject{航空宇航科学与技术}{Aeronautical and Astronautical Science and Technology}
\subject{航空工程}{Aeronautical Engineering}

% 答辩委员会委员 显示的顺序和这里的一样 第一个人是主席
% Committee member of your defence, notice that the order shows in the thesis is same as here, and the first one is the chairperson
% each member should be put as {Organization,Name,Title} split by comma
\addcommitteemember{西安交通大学,张长长,教授}
\addcommitteemember{西安理工大学,王旺旺,教授}
\addcommitteemember{国网陕西经济技术研究院,李力,高工}
\addcommitteemember{西安交通大学,东方不败,副教授}
\addcommitteemember{西安交通大学,赵照,研究员}

% 答辩时间(手动指定)
% Defence date, input manually
\defensedate{2021}{06}{22}

% 答辩地点(涉密论文请手动设置为「西安交通大学」)
% Defence location, default value is 「西安交通大学」
\defenseloc{西安交通大学主楼E座303室}

% 论文提交日期,不输入参数则默认使用当前日期,如手动指明年月,请在第一个可选参数内填写年份,第二个可选参数填写月份(均为阿拉伯数字)
% Submission date of this thesis, if you not put it manually, it will use the current time
% \submitdate[2021][06]
\submitdate

% 常规审阅人 要求和答辩委员会委员一样
% General reviewer list, same requirements as the committee member
\addgeneralreviewer{西安交通大学,张长长,教授}
\addgeneralreviewer{西安理工大学,王旺旺,教授}
\addgeneralreviewer{国网陕西经济技术研究院,李力,高工}
\addgeneralreviewer{西安交通大学,东方不败,副教授}
\addgeneralreviewer{西安交通大学,赵照,研究员}

% 学院
% School or Faculty, unused now
% \school{电气工程学院}{School of Electrical Engineering}

% 专业[学士学位使用]
% Major, unused now
% \stumajor{计算机科学与计数}{Computer Science}

% 学号[学士学位使用]
% Student ID, unused now
% \stuid{}

% 班级[学士学位使用]
% Administrative class, unused now
% \adminclass{电气7xx班}{}

% 参考文献源 参数中不要添加 .bib
% 请使用 \addreferenceresource 添加数据库(可导入多个参考文献数据库)
% 若自动化导入攻读学位期间的成果则使用 \addachivementresource
\addreferenceresource{References/reference}
\addreferenceresource{References/reference}
\addachivementresource{References/achievement}

%%%%%%%%%%%%%%%%%%%%%%%%%%%%%%%%%%%%%%%%%%%%
%% 如果有使用其他包,请在这里添加
%% If you need other packages, use them here

% \usepackage{}


%%%%%%%%%%%%%%%%%%%%%%%%%%%%%%%%%%%%%%%%%%%%%%%%%%%%%%%%%%%%%%%%%%%%%%%%%%%%%%%%%%%
%% 注意:根据校方要求,以下所有页面顺序不可调整
%% Notice: The order of these pages are defined in the requirements of the University.

%% 但在最终提交前,可以通过注释所有 \thesis 开始的命令设置是否生成各个部分
%% 或根据说明调整 latexmkrc 文件,使用 \includeonly 命令导入部分章节。

% \includeonly{
%   Main_Spine/c1,
%   Main_Spine/c2,
%   Main_Spine/c3,
%   Main_Spine/c4,
%   Main_Spine/c5,
%   Main_Spine/c6,
% }

\begin{document}
% [自动生成] 中英题名页
% [Auto Generate] Chinese English Title Page
\thesistitles

% [自动生成] 答辩委员会页
% [Auto Generate] Defense Committee Pages
\thesiscommittes

% 生成摘要页 修改 Main_Miscellaneous/abstract_chs/eng.tex 中的内容
% Abstract, Rewrite your content in Main_Miscellaneous/abstract_chs/eng.tex
\thesisabstract

% [自动生成] 中英目录页
% [Auto Generate] Table of Contents
\thesistableofcontens

% 主要符号表 修改 Main_Miscellaneous/glossary.tex 中的内容
% Glossaries Page, Rewrite your content in Main_Miscellaneous/glossary.tex
\thesisglossarylist

% 正文 注意英文正文写作时,中、英标题还是先中后英标题;同时,下面参数的顺序有意义,不要乱放
% Main contents, X of cX is the chapter of the thesis, you can change it if you want, the order matters
% and KEEP Chinese Title as the FIRST argument, English Titile as the SECOND, if you use `english' option

% 可以通过 \thesisbody 直接导入各部分正文,但也可通过 \thesisbodybegin & \include & \thesisbodyend 组合导入正文

% \thesisbody{
%     Main_Spine/c1,
%     Main_Spine/c2,
%     Main_Spine/c3,
%     Main_Spine/c4,
%     Main_Spine/c5,
%     Main_Spine/c6
% }

\thesisbodybegin
% !TeX root = ../main.tex
\chapter{绪论}
\echapter{Introductions}

\section{背景}
\esection{Backgrounds}

\zhlipsum[11]

\subsection{摸鱼的历史}
\esubsection{History of Underwork}

\zhlipsum[12]

\subsection{摸鱼的历史}
\esubsection{History of Underwork}


\subsubsection{第四级标题}

\paragraph{第五级标题}

\subparagraph{第六级标题}

\subsubparagraph{第七级标题}


公式如下:
\begin{equation}\label{eqn:c1:mdl:constraint_discharge}
    -e^{\max}_\text{dis} \leq a_t \leq e^{\max}_\text{ch}
\end{equation}
上式表示

所以如式\eqref{eqn:c1:mdl:constraint_discharge}所示:
\begin{equation}
    -e^{\max}_\text{dis} \leq a_t \leq e^{\max}_\text{ch}
\end{equation}

最后\footnote{脚注序号“\ding{172},……,\ding{180}”的字体是“正文”,不是“上标”,序号与脚注内容文字之间空1个半角字符,脚注的段落格式为:单倍行距,段前空0磅,段后空0磅,悬挂缩进1.5字符;中文用宋体,字号为小五号,英文和数字用Times New Roman字体,字号为9磅;中英文混排时,所有标点符号(例如逗号“,”、括号“()”等)一律使用中文输入状态下的标点符号,但小数点采用英文状态下的样式“.”。}

\begin{enumerate}
    \item 123
    \item 231421
    \item 124124
\end{enumerate}

\begin{theorem}[勾股定理]
    若 $a,b$ 为直角三角形的两条直角边,$c$ 为斜边,那么 $a^2 + b^2 + c^2.$
\end{theorem}

\begin{proof}
{
    通过...

    所以:
    \begin{equation*}
        G(x, y) = G(y, x).  \qedhere
    \end{equation*}
}
\end{proof}

\begin{proposition}
    所以:
    \begin{equation*}
        G(x, y) = G(y, x).
    \end{equation*}
\end{proposition}


\begin{conjecture}[勾股定理]
    若 $a,b$ 为直角三角形的两条直角边,$c$ 为斜边,那么 $a^2 + b^2 + c^2.$
\end{conjecture}

\begin{axiom}[勾股定理]
    若 $a,b$ 为直角三角形的两条直角边,$c$ 为斜边,那么 $a^2 + b^2 + c^2.$
\end{axiom}

\begin{definition}[勾股定理]
    若 $a,b$ 为直角三角形的两条直角边,$c$ 为斜边,那么 $a^2 + b^2 + c^2.$
\end{definition}





% !TeX root = ../main.tex
\xchapter{浮动体:图表}{Floating: Figures, Tables}

\xsection{图}{Figures}

\xsubsection{应该使用 subcaption 而不是 subfigure}{Use subcaption instead of subfigure}

\begin{verbatim}
    \begin{figure}[H]
        \begin{subfigure}[b]{0.49\linewidth}
            \centering
            \includegraphics[height=5.8cm]{xjtu_blue.pdf}
            \subcaption{蓝色校徽}
        \end{subfigure}
        \begin{subfigure}[b]{0.49\linewidth}
            \centering
            \includegraphics[height=6cm]{xjtu_gray.pdf}
            \subcaption{灰色校徽}
            \label{subfig:icon}
        \end{subfigure}
        \caption{校徽}
    \end{figure}
    
\end{verbatim}

\begin{figure}[H]
    \begin{subfigure}[b]{0.49\linewidth}
        \centering
        \includegraphics[height=5.8cm]{xjtu_blue.pdf}
        \subcaption{蓝色校徽}
    \end{subfigure}
    \begin{subfigure}[b]{0.49\linewidth}
        \centering
        \includegraphics[height=6cm]{xjtu_gray.pdf}
        \subcaption{灰色校徽}
        \label{subfig:icon}
    \end{subfigure}
    \caption{校徽}
\end{figure}


\xsection{表}{Tables}

表格要求采用三线表,与文字齐宽,顶线与底线线粗为 $1.5$ pt,中线线粗是 $1$ pt,如表 \ref{tab_ch2} 所示\footnote{{\color{red}注意}:图表中的变量与单位通过斜线 $/$ 隔开。}。

通栏表格应使用 tabularx 环境:

\begin{verbatim}
    \begin{table}[H]
        \caption{左对齐}
        \begin{tabularx}{\textwidth}{XX}
        \toprule
            \textbf{Symptom} & \textbf{Metric} \\
        \midrule
            Class that has many accessor methods and accesses a lot of external data & ATFD is more than a few\\
            Class that is large and complex & WMC is high\\
            Class that has a lot of methods that only operate on a proper subse&\\
        \bottomrule
        \end{tabularx}
    \end{table}
\end{verbatim}

\begin{table}[H]
    \caption{左对齐}
    \begin{tabularx}{\textwidth}{XX}
    \toprule
        \textbf{Symptom} & \textbf{Metric} \\
    \midrule
        Class that has many accessor methods and accesses a lot of external data & ATFD is more than a few\\
        Class that is large and complex & WMC is high\\
        Class that has a lot of methods that only operate on a proper subse & \\
    \bottomrule
    \end{tabularx}
\end{table}

\begin{table}[H]
    \caption{居中}
    \begin{tabularx}{\textwidth}{YY}
    \toprule
        \textbf{Symptom} & \textbf{Metric} \\
    \midrule
        Class  & ATFD \\
        Class  & WMC \\
        Class  & TCC \\
    \bottomrule
    \end{tabularx}
\end{table}

\begin{table}[!ht]
	\renewcommand{\arraystretch}{1.2}
	\centering\wuhao
	\caption{表题也是五号字} \label{tab_ch2} \vspace{2mm}
	\begin{tabularx}{\textwidth}{*{4}Y}
	\toprule[1.5pt]
		Interference & DOA / degree & Bandwidth / MHz & INR / dB \\
	\midrule[1pt]
		1 & $-30$ & 20 & 60 \\
		2 & 20 & 10 & 50 \\
		3 & 40 & 5 & 40 \\
	\bottomrule[1.5pt]
	\end{tabularx}
\end{table}
% !TeX root = ../main.tex
\chapter{摸鱼的算法}
\echapter{Algorithm of Underwork}


\begin{algorithm}[H]
	\caption{Event Detection}
	\label{algo:event1}
	\LinesNumbered
	\KwIn{FFT Bins $B_{1}, B_{2}, B_{3}, ..., B_{n}$}
	\KwOut{Event start point $S$, end point $E$}
	\For{i=1:n}{	
		\eIf{max($B_{i}$) : max($B_{(i+4)}$)$>$-80}{
			$S$=i; \tcp*{Start Point}
		\For{j=S:n}{
			\eIf{max($B_{j}$) : max($B_{(j+4)}$)$<$-100}{
				$E$=j; \tcp*{End Point}
				Return [$S,E$];
			}{j++;}
		}
	}{
	i++;}
}
\end{algorithm}

\lstinputlisting[style=sty_basic,language=matlab,caption={test algorithm}]{pso.m}



\subsection{摸鱼的历史}
\esubsection{History of Underwork}

\subsection{摸鱼的历史}
\esubsection{History of Underwork}

\zhlipsum[12,7]

\subsection{摸鱼的历史}
\esubsection{History of Underwork}

\subsection{摸鱼的历史}
\esubsection{History of Underwork}

\blindtext

\blindtext

\subsection{摸鱼的历史}
\esubsection{History of Underwork}

\subsection{摸鱼的历史}
\esubsection{History of Underwork}
% !TeX root = ../main.tex

\xchapter{算法与代码}{Algorithm and Code}

\xsection{算法展示}{Algorithm}

\begin{algorithm}[H]
	\caption{Event Detection}
	\label{algo:event1}
	\LinesNumbered
	\KwIn{FFT Bins $B_{1}, B_{2}, B_{3}, ..., B_{n}$}
	\KwOut{Event start point $S$, end point $E$}
	\For{i=1:n}{	
		\eIf{max($B_{i}$) : max($B_{(i+4)}$)$>$-80}{
			$S$=i; \tcp*{Start Point}
		\For{j=S:n}{
			\eIf{max($B_{j}$) : max($B_{(j+4)}$)$<$-100}{
				$E$=j; \tcp*{End Point}
				Return [$S,E$];

				Return [$S,E$];

				Return [$S,E$];

				Return [$S,E$];

				Return [$S,E$];

				Return [$S,E$];

				Return [$S,E$];

				Return [$S,E$];

				Return [$S,E$];

				Return [$S,E$];

				Return [$S,E$];

				Return [$S,E$];

				Return [$S,E$];

				Return [$S,E$];

				Return [$S,E$];

				Return [$S,E$];

				Return [$S,E$];

				Return [$S,E$];

				Return [$S,E$];

				Return [$S,E$];

				Return [$S,E$];

				Return [$S,E$];

				Return [$S,E$];

				Return [$S,E$];

				Return [$S,E$];

				Return [$S,E$];

				Return [$S,E$];

				Return [$S,E$];

				Return [$S,E$];

				Return [$S,E$];

				Return [$S,E$];

				Return [$S,E$];

				Return [$S,E$];

			}{j++;}
		}
	}{
	i++;}
}
\end{algorithm}


\xsection{导入代码}{Input codes}

\begin{verbatim}
    \lstinputlisting[style=sty_basic,
                    language=matlab,
                    caption={test algorithm}]{pso.m}
\end{verbatim}

\lstinputlisting[style=sty_basic,
                    language=matlab,
                    caption={test algorithm}]{pso.m}
% !TeX root = ../main.tex

\xchapter{本模板已经使用的 Packages}{Loaded Packages}
% !TeX root = ../main.tex

\xchapter{本模板已载入的 Packages}{Loaded Packages}

\begin{table}[h]
  \begin{tblr}{
    colspec={*{8}{X[c,m]}},
    vlines, hlines,
    leftsep=0pt,rightsep=0pt,
  }
  expl3     & ifxetex    & xcolor   & tikz        & tikz-3dplot & zhnumber    & datetime2  & indentfirst \\
  setspace  & etoolbox   & xpatch   & xparse      & calc        & ulem        & ifthen     & realboxes   \\
  blindtext & zhlipsum   & amsmath  & unicode-math     & amsthm      & thmtools    & glossaries & upgreek     \\
  pifont    & array      & commath  & siunitx     & mathtools   & -    & geometry   & ifoddpage   \\
  emptypage & pdfpages   & xeCJK    & fontenc     & lmodern     & anyfontsize & mathrsfs   & amsfonts    \\
  ctex      & ifplatform & graphicx & algorithm2e & tcolorbox   & listings    & enumitem   & footmisc    \\
  fancyvrb  & fancyhdr   & titlesec & float       & newfloat    & tabularray    & -  & -        \\
  -  & -  & - & caption     & subcaption  & xurl        & hyperref   & bookmark    \\
  titletoc  & tocloft    & biblatex & appendix    & cleveref    &             &            &             \\
  \end{tblr}
\end{table}
\thesisbodyend

% 致谢 修改 Main_Miscellaneous/acknowledegment.tex 中的内容
% Acknowledgement, Rewrite your content in Main_Miscellaneous/acknowledegment.tex
\thesisacknowledegment

% [自动生成] 参考文献 默认使用 References/reference.bib,手动指定请在导言区 \addbibresource 处指定
% [Auto Generate] Bibliography, Default file is References/reference.bib, change argument in \addbibresource for manual specification
\thesisbibliography

% 附录(有几个附录就导入几个文件(不加.tex后缀)),
% Appendi(x/ies), argument should not have the .tex suffix
\thesisappendix{Main_Miscellaneous/appendix_a,
                Main_Miscellaneous/appendix_b}

% 攻读学位期间取得的研究成果
% 添加 [auto] 参数则读取通过 \addachivementresource 添加的数据库,否则读取 Main_Miscellaneous/achievement.tex 的内容 注意盲审时需要手动修改格式
% 自动读取的数据库中 若条目含有 AUTHOR+an = {X=highlight} 则第 X 位作者会被加粗
% Achievement, argument [auto] will load data from database added by \addachivementresource, or the data from Main_Miscellaneous/achievement.tex
% \thesisachivements
\thesisachivements[auto]

% 答辩委员会决议 修改 Main_Miscellaneous/decision.tex 中的内容
% Decision, Rewrite your content in Main_Miscellaneous/decision.tex
\thesisdecision

% [自动生成] 常规评阅人名单 需要手动指定两个数字作为'本学位论文共接受 {#1} 位专家评阅,其中常规评阅人 {2}名'内容参数
% [Auto Generate] Reviewer List, set two number as the content in this page 
\thesisreviewers{7}{5}

% [自动生成] 独创性声明
% [Auto Generate] Originality Declaration
\thesisdeclarations

\end{document}