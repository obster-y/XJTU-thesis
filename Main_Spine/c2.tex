% !TeX root = ../main.tex
\xchapter{浮动体:图表}{Floating: Figures, Tables}

\xsection{图}{Figures}

\xsubsection{应该使用 subcaption 而不是 subfigure}{Use subcaption instead of subfigure}

\begin{verbatim}
    \begin{figure}[H]
        \begin{subfigure}[b]{0.49\linewidth}
            \centering
            \includegraphics[height=5.8cm]{xjtu_blue.pdf}
            \subcaption{蓝色校徽}
        \end{subfigure}
        \begin{subfigure}[b]{0.49\linewidth}
            \centering
            \includegraphics[height=6cm]{xjtu_gray.pdf}
            \subcaption{灰色校徽}
            \label{subfig:icon}
        \end{subfigure}
        \caption{校徽}
    \end{figure}
    
\end{verbatim}

\begin{figure}[H]
    \begin{subfigure}[b]{0.49\linewidth}
        \centering
        \includegraphics[height=5.8cm]{xjtu_blue.pdf}
        \subcaption{蓝色校徽}
    \end{subfigure}
    \begin{subfigure}[b]{0.49\linewidth}
        \centering
        \includegraphics[height=6cm]{xjtu_gray.pdf}
        \subcaption{灰色校徽}
        \label{subfig:icon}
    \end{subfigure}
    \caption{校徽}
\end{figure}


\xsection{表}{Tables}

表格要求采用三线表,与文字齐宽,顶线与底线线粗为 $1.5$ pt,中线线粗是 $1$ pt,如表 \ref{tab_ch2} 所示\footnote{{\color{red}注意}:图表中的变量与单位通过斜线 $/$ 隔开。}。

通栏表格应使用 tabularx 环境:

\begin{verbatim}
    \begin{table}[H]
        \caption{左对齐}
        \begin{tabularx}{\textwidth}{XX}
        \toprule
            \textbf{Symptom} & \textbf{Metric} \\
        \midrule
            Class that has many accessor methods and accesses a lot of external data & ATFD is more than a few\\
            Class that is large and complex & WMC is high\\
            Class that has a lot of methods that only operate on a proper subse&\\
        \bottomrule
        \end{tabularx}
    \end{table}
\end{verbatim}

\begin{table}[H]
    \caption{左对齐}
    \begin{tabularx}{\textwidth}{XX}
    \toprule
        \textbf{Symptom} & \textbf{Metric} \\
    \midrule
        Class that has many accessor methods and accesses a lot of external data & ATFD is more than a few\\
        Class that is large and complex & WMC is high\\
        Class that has a lot of methods that only operate on a proper subse & \\
    \bottomrule
    \end{tabularx}
\end{table}

\begin{table}[H]
    \caption{居中}
    \begin{tabularx}{\textwidth}{YY}
    \toprule
        \textbf{Symptom} & \textbf{Metric} \\
    \midrule
        Class  & ATFD \\
        Class  & WMC \\
        Class  & TCC \\
    \bottomrule
    \end{tabularx}
\end{table}

\begin{table}[!ht]
	\renewcommand{\arraystretch}{1.2}
	\centering\wuhao
	\caption{表题也是五号字} \label{tab_ch2} \vspace{2mm}
	\begin{tabularx}{\textwidth}{*{4}Y}
	\toprule[1.5pt]
		Interference & DOA / degree & Bandwidth / MHz & INR / dB \\
	\midrule[1pt]
		1 & $-30$ & 20 & 60 \\
		2 & 20 & 10 & 50 \\
		3 & 40 & 5 & 40 \\
	\bottomrule[1.5pt]
	\end{tabularx}
\end{table}