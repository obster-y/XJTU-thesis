% !TeX root = ../main.tex

\xchapter{浮动体:图表}{Floating: Figures, Tables}

请先注意,在\LaTeX 中,图表不一定要像 Word 一样被固定在某个区域,程序会自动根据上下文为图片选择合适的位置,同时还有具有超链接的交叉引用自动定位图片位置,因此不要强求图片或表格一定处于某个位置。如非要定位,则应在浮动体环境中使用\clist{[H]}选项指定位置。

\begin{tcolorbox}[colback=red!5!white,colframe=red!75!black]
  \begin{enumerate}[leftmargin=0.5cm]
    \item 图、表、公式等一律用阿拉伯数字分章连续编号,如 图1-3、表2-1、(3-2)等。图、表、公式等与正文之间间隔0.5行。

    \item 图应有图题,表应有表题,并分别置于图号和表号之后,图号和图题应置于图下方的居中位置,表号和表题应置于表上方的居中位置。引用图或表应在图题或表题右上角标出文献来源。

    \item 若图或表中有附注,采用英文小写字母顺序编号,附注写在图或表的下方。
  \end{enumerate}
\end{tcolorbox}

\xsection{图}{Figures}


\begin{tcolorbox}[colback=red!5!white,colframe=red!75!black]
  \begin{enumerate}[leftmargin=0.5cm]
    \item 插图须紧跟文述。在正文中,一般应先见图号及图的内容后再见图,一般情况下不能提前见图,特殊情况须延后的插图不应跨节;
    \item 提供照片应大小适宜,主题明确,层次清楚,金相照片一定要有比例尺;
    \item 图应具有“自明性”,即只看图、图题和图例,不阅读正文,就可理解图意。图中的标目是说明坐标轴物理意义的项目,它是由物理量的符号或名称和相应的单位组成。物理量的符号由斜体字母标注,单位的符号使用正体字母标注,量与单位间用斜线隔开。例如:$I/\SI{}{A}$,$\rho/\SI{}{kg\cdot m^{-3}}$,$F/\SI{}{N}$,$v/\SI{}{m\cdot s^{-1}}$等等;
    \item 图中用字为五号,如排列过密,用五号字有困难时,可小于五号字,但不得小于七号字;
    \item 图尺寸的\textbf{一般宽高比}应为 \SI{6.67}{cm}$\times$\SI{5.00}{cm}。特殊情况下,也可为 \SI{9.00}{cm}$\times$\SI{6.75}{cm},或 \SI{13.5}{cm}$\times$\SI{9.00}{cm}。总之,一篇论文中,同类图片的大小应该一致,编排美观、整齐;
    \item 一幅图如有若干幅\textbf{分图},均应编分图号,用(a),(b),(c),...... 按顺序编排,且各分图的分题注直接列在各自分图的正下方,总题注列在所有分图的下方正中「
  \end{enumerate}
\end{tcolorbox}

\xsubsection{单张图片的使用}{Single Figure}

\begin{texcodeonly}[]{}
  \begin{figure}[h]
    \centering
    \includegraphics[height=5.8cm]{xjtu_blue.pdf}
    \caption{校徽}
  \end{figure}
\end{texcodeonly}

\begin{figure}[h]
  \centering
  \includegraphics[height=5.8cm]{xjtu_blue.pdf}
  \caption{校徽}
\end{figure}

一些图片下方的文字。一些图片下方的文字。一些图片下方的文字。一些图片下方的文字。一些图片下方的文字。一些图片下方的文字。一些图片下方的文字。一些图片下方的文字。一些图片下方的文字。一些图片下方的文字。一些图片下方的文字。一些图片下方的文字。一些图片下方的文字。

\xsubsection{多张图片:应该使用 subcaption 而不是 subfigure}{Use subcaption instead of subfigure}

\begin{texcodeonly}[]{}
  \begin{figure}[h]
    \begin{subfigure}[b]{0.49\linewidth}
        \centering
        \includegraphics[height=5.8cm]{xjtu_blue.pdf}
        \subcaption{蓝色校徽}
    \end{subfigure}
    \begin{subfigure}[b]{0.49\linewidth}
        \centering
        \includegraphics[height=5.8cm]{xjtu_gray.pdf}
        \subcaption{灰色校徽}
        \label{subfigure:icon}
    \end{subfigure}
    \caption{校徽}
    \label{fig:icon}
  \end{figure}
\end{texcodeonly}

\begin{figure}[h]
  \begin{subfigure}[b]{0.49\linewidth}
      \centering
      \includegraphics[height=5.8cm]{xjtu_blue.pdf}
      \subcaption{蓝色校徽}
  \end{subfigure}
  \begin{subfigure}[b]{0.49\linewidth}
      \centering
      \includegraphics[height=5.8cm]{xjtu_gray.pdf}
      \subcaption{灰色校徽}
      \label{subfigure:icon}
  \end{subfigure}
  \caption{校徽}
  \label{figure:icon}
\end{figure}

\xsubsection{多张子图分页}{Break multiple figures}

\newcommand{\subfigg}{
  \begin{subfigure}[b]{0.49\linewidth}
    \centering
    \includegraphics[height=4cm]{xjtu_blue.pdf}
    \subcaption{蓝色校徽}
  \end{subfigure}
}

为了简化叙述,定义一个命令 \clist{\subfigg} 代表插入一个子图:
\begin{texcodeonly}[]{}
  \newcommand{\subfigg}{
    \begin{subfigure}[b]{0.49\linewidth}
      \centering
      \includegraphics[height=4cm]{xjtu_blue.pdf}
      \subcaption{蓝色校徽}
    \end{subfigure}
  }
\end{texcodeonly}

在\clist[columbiablue]{subfig}宏包中提供了\clist{\ContinuedFloat}命令,使一个浮动体环境与其内子图的编号跟随前一个子图,此时两个浮动体环境可以分别设置标题,若具有相同的标题,或后续图的标题设置为 \clist{(续)},则可以实现子图分页的效果。同时,模板提供了 \clist{\floatcontinue} ,通过包装\clist{\ContinuedFloat}使得子图在此命令处断页,不产生图标题,并在下一页生成图题。此两种命令均可使用,按照喜欢或需求使用。
\clearpage
\subsubsection{使用 ContinuedFloat}

\begin{texcodeonly}[]{}
  \begin{figure}[h]
    \subfigg    \subfigg
    \caption{校徽}
  \end{figure}
  
  \begin{figure}[H]
    \ContinuedFloat
    \subfigg    \subfigg
    \subfigg    \subfigg
    \caption{校徽} % 或 \caption{(续)}
  \end{figure}
\end{texcodeonly}

\begin{figure}[h]
  \subfigg    \subfigg
  \caption{校徽}
\end{figure}

\begin{figure}[H]
  \ContinuedFloat
  \subfigg    \subfigg
  \subfigg    \subfigg
  \caption{校徽} % 或 \caption{(续)}
\end{figure}

\subsubsection{使用 floatcontinue}

\begin{texcodeonly}[]{}
  \begin{figure}[h]
    \subfigg    \subfigg
    \floatcontinue{tb}
    \subfigg    \subfigg
    \subfigg    \subfigg
    \caption{校徽}
  \end{figure}
\end{texcodeonly}

\begin{figure}[h]
  \subfigg    \subfigg
  \floatcontinue{tb}
  \subfigg    \subfigg
  \subfigg    \subfigg
  \caption{校徽}
\end{figure}


\clearpage

\xsection{表}{Tables}

\begin{tcolorbox}[colback=red!5!white,colframe=red!75!black]
  \begin{enumerate}[leftmargin=0.5cm]
    \item 如某个表需要转页接排,在随后的各页上应重复表的编号。编号后跟表题(可省略)和“(续)”,如表 1-1(续),续表均应重复表头和关于单位的陈述。表格的设计应紧跟文述。表的编排一般是内容和测试项目由左至右横读,数据依序竖读,应有自明性。若为大表或作为工具使用的表格,可作为附表在附录中给出,论文中的表格参数应标明量和单位的符号;
    \item 表中各物理量及量纲均按国际标准(SI) 及国家规定的法定符号和法定计量单位标注;
    \item 一律使用三线表,与文字齐宽,顶线与底线线粗为 $1.5$ pt,中线线粗是 $1$ pt;
    \item 使用他人表格须注明出处。
    \item 表中用字为五号字体。如排列过密,用五号字有困难时,可小于五号字,但不小于七号。
    \item 表格必须通栏,即表格宽度与正文版面平齐。
  \end{enumerate}
\end{tcolorbox}


\xsubsection{普通表格}{Normal Tables}


考虑到使用成本、内容与样式分离。模板默认调用现代化的表格绘制宏包 \clist[columbiablue]{tabularray}。该宏包的使用方法请使用 \clist{texdoc tabularray},网上也有此手册的中文翻译。请确保使用的发行版包含这一宏包(即更新至 2023 及以后的版本)。

非跨页表格应使用 tblr 环境制作表格,使用 figure 作为浮动体环境,使用 \clist{\toprule},\clist{\midrule} 和 \clist{\bottomrule} 指定三线表的三条线:

\begin{texcode}[]{}
    \begin{table}[H]
        \caption{tblr 基本用法}
        \label{tbl:2-1}
        \begin{tblr}{
            % colspec={X[r,m,2]X[c,3]X[l,1]}
            X[r,m,2]X[c,3]X[l,1]
        }
            \toprule
            Head & Test & 测试 \\
            \midrule
            Class that has many accessor methods and accesses a lot of external data & ATFD is more than a few & 3 \\
            1 & 2 & 3 \\
            1 & 2 & 3 \\
            \bottomrule
        \end{tblr}
    \end{table}
\end{texcode}

tblr 环境各列的对齐方式可以以键值对的方式在必选参数中设置,或者直接作为唯一必选参数(参考 tabularray 手册第 24 页)。列类型包括 c/l/r, t/b/m, X 等。c/l/r 为默认的三种对齐方式,t/b/m 指定单元格内容超过列宽时垂直方向的对齐方式, X 类型会自动调整列宽使全表宽度与页宽一致,X 类型列的水平、垂直对齐方式以类似 X[c] 的方式在 X 后的方括号内指定。X 类型后的方括号可以指定此列宽度与其他X类列宽度的相对大小,也可使用 \clist{wd=A}直接指定列宽。推荐使用 X 作为列类型。

\begin{texcode}[]{}
    \begin{tblr}{
        colspec={|ll|X[c]|rr|},
        cell{1}{1}={r=3,c=2}{r,m},
        cell{1}{3}={r=2,c=3}{l,t},
        cell{3}{Z}={}{red},
        hline{4-6}={2pt,blue},
    }
        \hline
        1-1 & 1-2 & 1-3 & 1-4 & 1-5 \\
        \hline
        2-1 & 2-2 & 2-3 & 2-4 & 2-5 \\
        \hline
        3-1 & 3-2 & MIDDLE & 3-4 & 3-5 \\
        % \hline (4th)
        4-1 & 4-2 & 4-3 & 4-4 & 4-5 \\
        % \hline (5th)
        5-1 & 5-2 & 5-3 & 5-4 & 5-5 \\
        % \hline (6th)
    \end{tblr}
\end{texcode}

tabularray 的优点在于格式和内容分离,使用更贴合人类习惯。单元格的合并,文字、背景、边框的样式都可以放在参数中设置。 \clist{cell{R}{C}={merge}{style}}用于设置单元格合并,其中 R,C 指定合并单元格左上角的位置,merge参数指定合并多少行多少列,style 调整合并后的单元格样式,其中 R,C 除了使用行号进行索引,可以使用 X,Y,Z 索引倒数第三、二、一行/列,可以使用 odd/even 选择 奇/偶列,还可以使用连字符(dash)选择区间,非常方便。

\xsubsection{复杂表格}{Complicated tables}

在三线表中可以加辅助线,以适应较复杂表格的需要,具体使用方法请自行阅读宏包:

\begin{texcode}[]{}
\begin{table}[H]
  \centering
  \caption{compare with other approachs}
  \label{table:methodcompare}
  \begin{tblr}{
      colspec={*{7}{X[c]}},
      cell{1}{1}={r=2}{c,m},
      cell{1}{2,5}={c=3}{c,m},
      hline{2,4,6}={1pt},
      hline{1,3,Z}={1.5pt},
      row{4,5}={rowsep=0pt},
      row{Z}={m,font=\bfseries},
  }
    Model & trigger inditification & & & Event Extraction & & \\
    & P(\%) & R(\%) & F1(\%) & P(\%) & R(\%) & F1(\%) \\
    Baseline1 & 76.84 & 76.84 & 76.84 & 76.84 & 76.84 & 76.84 \\
    Baseline2  & 76.84 & 76.84 & 76.84 & 76.84 & 76.84 & 76.84 \\
    Baseline3  & 76.84 & 76.84 & 76.84 & 76.84 & 76.84 & 76.84 \\
    Our approach  & 76.84 & 76.84 & 76.84 & 76.84 & 76.84 & 76.84 \\
  \end{tblr}
\end{table}
\end{texcode}

\xsubsection{长表格}{Long Tables}

当表格过长时,使用\clist{longtblr}环境,此时不需要浮动体环境。在可选参数中需要设置\clist{theme=xjtu_std},以调用预先设置好的续表样式,同时设置表格标题、label、脚注等,具体使用方法请自行阅读宏包。

\begin{texcodeonly}[]{}
    \begin{longtblr}[
        theme = xjtu_std,
        caption = {A Long Long Long Long Long Long Long Table },
        entry = {Short Caption},
        label = {tblr:test},
        note{a} = {脚注},
        note{$\dag$} = {还是脚注},
        remark{Note} = {注释,即表格内没有标注符号的脚注},
        remark{Source} = {还是注释},
    ]{
        colspec = {XX[c]X},
        rowhead = 1, rowfoot = 0,
    }
    \toprule
    Head & Head & Head \\
    \midrule
    Epsilon & Zeta\TblrNote{a} & Eta \\
    Iota & Kappa\TblrNote{$\dag$} & Lambda \\
    Nu & Xi & Omicron \\
    \dots 
    Foot & Foot & Foot \\
    \bottomrule
\end{texcodeonly}

\begin{longtblr}[
        theme = xjtu_std,
        caption = {A Long Long Long Long Long Long Long Table },
        entry = {Short Caption},
        label = {tblr:test},
        note{a} = {脚注},
        note{$\dag$} = {还是脚注},
        remark{Note} = {注释,即表格内没有标注符号的脚注},
        remark{Source} = {还是注释},
    ]{
        colspec = {XX[c]X},
        rowhead = 1, rowfoot = 0,
    }
    \toprule
    Head & Head & Head \\
    \midrule
    Head & Head & Head \\
    Alpha & Beta & Gamma \\
    Epsilon & Zeta\TblrNote{a} & Eta \\
    Iota & Kappa\TblrNote{$\dag$} & Lambda \\
    Nu & Xi & Omicron \\
    Rho & Sigma & Tau \\
    Phi & Chi & Psi \\
    %
    % ......
    Alpha & Beta & Gamma \\
    Epsilon & Zeta & Eta \\
    Iota & Kappa & Lambda \\
    Nu & Xi & Omicron \\
    Rho & Sigma & Tau \\
    Phi & Chi & Psi \\
    % Foot & Foot & Foot \\
    Alpha & Beta & Gamma \\
    \midrule
    Epsilon & Zeta & Eta \\
    Iota & Kappa & Lambda \\
    Nu & Xi & Omicron \\
    Rho & Sigma & Tau \\
    Phi & Chi & Psi \\
    Foot & Foot & Foot \\
    Alpha & Beta & Gamma \\
    Epsilon & Zeta & Eta \\
    Iota & Kappa & Lambda \\
    Nu & Xi & Omicron \\
    Rho & Sigma & Tau \\
    Phi & Chi & Psi \\
    Foot & Foot & Foot \\
    Alpha & Beta & Gamma \\
    Epsilon & Zeta & Eta \\
    Iota & Kappa & Lambda \\
    Nu & Xi & Omicron \\
    Rho & Sigma & Tau \\
    Phi & Chi & Psi \\
    Foot & Foot & Foot \\
    \bottomrule
\end{longtblr}
