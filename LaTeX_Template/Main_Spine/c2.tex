% !TeX root = ../main.tex

\xchapter{浮动体:图表}{Floating: Figures, Tables}

请先注意,在\LaTeX 中,图表不一定要像 Word 一样被固定在某个区域,程序会自动根据上下文为图片选择合适的位置,同时还有具有超链接的交叉引用自动定位图片位置,因此不要强求图片或表格一定处于某个位置。如非要定位,则应在浮动体环境中使用\clist{[H]}选项指定位置。

\begin{tcolorbox}[colback=red!5!white,colframe=red!75!black]
  \begin{enumerate}[leftmargin=0.5cm]
    \item 图、表、公式等一律用阿拉伯数字分章连续编号,如 图1-3、表2-1、(3-2)等。图、表、公式等与正文之间间隔0.5行。

    \item 图应有图题,表应有表题,并分别置于图号和表号之后,图号和图题应置于图下方的居中位置,表号和表题应置于表上方的居中位置。引用图或表应在图题或表题右上角标出文献来源。

    \item 若图或表中有附注,采用英文小写字母顺序编号,附注写在图或表的下方。
  \end{enumerate}
\end{tcolorbox}

\xsection{图}{Figures}


\begin{tcolorbox}[colback=red!5!white,colframe=red!75!black]
  \begin{enumerate}[leftmargin=0.5cm]
    \item 插图须紧跟文述。在正文中,一般应先见图号及图的内容后再见图,一般情况下不能提前见图,特殊情况须延后的插图不应跨节;
    \item 提供照片应大小适宜,主题明确,层次清楚,金相照片一定要有比例尺;
    \item 图应具有“自明性”,即只看图、图题和图例,不阅读正文,就可理解图意。图中的标目是说明坐标轴物理意义的项目,它是由物理量的符号或名称和相应的单位组成。物理量的符号由斜体字母标注,单位的符号使用正体字母标注,量与单位间用斜线隔开。例如:I/A,ρ/kg·m-3 ,F/N,υ/m·s-1等等;
    \item 图中用字为五号,如排列过密,用五号字有困难时,可小于五号字,但不得小于七号字;
    \item 图尺寸的\textbf{一般宽高比}应为 \SI{6.67}{cm}$\times$\SI{5.00}{cm}。特殊情况下,也可为 \SI{9.00}{cm}$\times$\SI{6.75}{cm},或 \SI{13.5}{cm}$\times$\SI{9.00}{cm}。总之,一篇论文中,同类图片的大小应该一致,编排美观、整齐;
    \item 一幅图如有若干幅\textbf{分图},均应编分图号,用(a),(b),(c),...... 按顺序编排,且各分图的分题注直接列在各自分图的正下方,总题注列在所有分图的下方正中「
  \end{enumerate}
\end{tcolorbox}

\xsubsection{单张图片的使用}{Single Figure}

\begin{texcodeonly}[]{}
  \begin{figure}[H]
    \centering
    \includegraphics[height=5.8cm]{xjtu_blue.pdf}
    \caption{校徽}
  \end{figure}
\end{texcodeonly}

\begin{figure}[H]
  \centering
  \includegraphics[height=5.8cm]{xjtu_blue.pdf}
  \caption{校徽}
\end{figure}


\xsubsection{多张图片:应该使用 subcaption 而不是 subfigure}{Use subcaption instead of subfigure}

\begin{texcodeonly}[]{}
  \begin{figure}[H]
    \begin{subfigure}[b]{0.49\linewidth}
        \centering
        \includegraphics[height=5.8cm]{xjtu_blue.pdf}
        \subcaption{蓝色校徽}
    \end{subfigure}
    \begin{subfigure}[b]{0.49\linewidth}
        \centering
        \includegraphics[height=5.8cm]{xjtu_gray.pdf}
        \subcaption{灰色校徽}
        \label{subfig:icon}
    \end{subfigure}
    \caption{校徽}
  \end{figure}
\end{texcodeonly}

\begin{figure}[H]
  \begin{subfigure}[b]{0.49\linewidth}
      \centering
      \includegraphics[height=5.8cm]{xjtu_blue.pdf}
      \subcaption{蓝色校徽}
  \end{subfigure}
  \begin{subfigure}[b]{0.49\linewidth}
      \centering
      \includegraphics[height=5.8cm]{xjtu_gray.pdf}
      \subcaption{灰色校徽}
      \label{subfig:icon}
  \end{subfigure}
  \caption{校徽}
\end{figure}

\xsubsection{多张子图分页}{Break multiple figures}

\newcommand{\subfigg}{
  \begin{subfigure}[b]{0.49\linewidth}
    \centering
    \includegraphics[height=5.8cm]{xjtu_blue.pdf}
    \subcaption{蓝色校徽}
  \end{subfigure}
}

\begin{texcodeonly}[]{}
  \begin{figure}[H]
    \subfigg    \subfigg
    \floatcontinue{tb}
    \subfigg    \subfigg
    \subfigg    \subfigg
    \caption{校徽}
  \end{figure}
\end{texcodeonly}

\begin{figure}[h]
  \subfigg    \subfigg
  \floatcontinue{tb}
  \subfigg    \subfigg
  \subfigg    \subfigg
  \caption{校徽}
\end{figure}

\clearpage

\xsection{表}{Tables}

\begin{tcolorbox}[colback=red!5!white,colframe=red!75!black]
  \begin{enumerate}[leftmargin=0.5cm]
    \item 如某个表需要转页接排,在随后的各页上应重复表的编号。编号后跟表题(可省略)和“(续)”,如表 1-1(续),续表均应重复表头和关于单位的陈述。表格的设计应紧跟文述。表的编排一般是内容和测试项目由左至右横读,数据依序竖读,应有自明性。若为大表或作为工具使用的表格,可作为附表在附录中给出,论文中的表格参数应标明量和单位的符号;
    \item 表中各物理量及量纲均按国际标准(SI) 及国家规定的法定符号和法定计量单位标注;
    \item 一律使用三线表,与文字齐宽,顶线与底线线粗为 $1.5$ pt,中线线粗是 $1$ pt;
    \item 使用他人表格须注明出处。
    \item 表中用字为五号字体。如排列过密,用五号字有困难时,可小于五号字,但不小于七号。
    \item 表格必须通栏,即表格宽度与正文版面平齐。
  \end{enumerate}
\end{tcolorbox}


\xsubsection{普通表格}{Normal Tables}


通栏表格应使用 tabularx 环境,其中 \clist{X,Y,Z} 分别对应初始表格的 \clist{l,c,r}

\begin{texcode}[]{}
  \begin{table}[H]
    \caption{左对齐}
    \begin{tabularx}{\textwidth}{XX}
    \toprule
        \textbf{Symptom} & \textbf{Metric} \\
    \midrule
        Class that has many accessor methods and accesses a lot of external data & ATFD is more than a few\\
        Class that is large and complex & WMC is high \\
        Class that has a lot of methods that only operate on a proper & \\
    \bottomrule
    \end{tabularx}
  \end{table}
\end{texcode}

\begin{texcode}[]{}
  \begin{table}[H]
    \caption{居中}
    \begin{tabularx}{\textwidth}{YY}
    \toprule
        \textbf{Symptom} & \textbf{Metric} \\
    \midrule
        Class  & ATFD \\
        Class  & WMC \\
        Class  & TCC \\
    \bottomrule
    \end{tabularx}
  \end{table}
\end{texcode}


\clearpage

\xsubsection{复杂表格}{Complicated tables}

在三线表中可以加辅助线,以适应较复杂表格的需要:

\begin{texcode}[]{}
\begin{table}[H]  
  \centering
  \caption{compare with other approachs}
  \label{tab:methodcompare}
  \begin{tabularx}{\textwidth}{*{7}Y}
    \toprule
    \multirow{2}*{Model} & \multicolumn{3}{c}{trigger identification} &  \multicolumn{3}{c}{Event Extraction} \\ 
    \cline{2-7}
    & P(\%) & R(\%) & F1(\%) & P(\%) & R(\%) & F1(\%) \\
    \midrule 
    Baseline1 & 76.84 & 76.84 & 76.84 & 76.84 & 76.84 & 76.84 \\
    \midrule[0.5pt]
    Baseline2  & 76.84 & 76.84 & 76.84 & 76.84 & 76.84 & 76.84 \\
    Baseline3  & 76.84 & 76.84 & 76.84 & 76.84 & 76.84 & 76.84 \\
    \midrule[0.5pt]
    {\bf Our approach}  & {\bf 76.84} & {\bf 76.84} & {\bf 76.84} & {\bf 76.84} & {\bf 76.84} & {\bf 76.84} \\
    \bottomrule
  \end{tabularx}
\end{table}
\end{texcode}

\clearpage

\xsubsection{长表格}{Long Tables}

当表格过长时,使用\clist{xltabular}环境,可以自动生成续表:

\begin{texcodeonly}[]{}
  \begin{xltabular}{\textwidth}{XX}
    \caption{左对齐} \\

      \toprule
      \textbf{Symptom} & \textbf{Metric} \\
      \midrule
    \endfirsthead % 以上设置第一张表表头

      \multicolumn{2}{c}{\tablename\ \thetable{}(续)} \\
      \toprule
      \textbf{Symptom} & \textbf{Metric} \\
      \midrule
    \endhead % 以上设置所有续表表头

      \bottomrule
      \multicolumn{2}{r}{表格待续} \\
    \endfoot % 以上设置第一张表结尾

      \bottomrule
    \endlastfoot % 以上设置最后一张表结尾

    Class that has many accessor methods and accesses a lot of external data & ATFD is more than a few \\
    Class that is large and complex & WMC is high \\
    Class that has a lot of methods that only operate on a proper & \\
    Class that has a lot of methods that only operate on a proper & \\
    Class that has a lot of methods that only operate on a proper & \\
    Class that has a lot of methods that only operate on a proper & \\
    Class that has a lot of methods that only operate on a proper & \\
    Class that has a lot of methods that only operate on a proper & \\
    Class that has a lot of methods that only operate on a proper & \\
    Class that has a lot of methods that only operate on a proper & \\
    Class that has a lot of methods that only operate on a proper & \\
    Class that has a lot of methods that only operate on a proper & \\
    Class that has a lot of methods that only operate on a proper & \\
    Class that has a lot of methods that only operate on a proper & \\
    Class that has a lot of methods that only operate on a proper & \\
    Class that has a lot of methods that only operate on a proper & \\
    Class that has a lot of methods that only operate on a proper & \\
    Class that has a lot of methods that only operate on a proper & \\
    Class that has a lot of methods that only operate on a proper & \\
    Class that has a lot of methods that only operate on a proper & \\
    Class that has a lot of methods that only operate on a proper & \\
  \end{xltabular}
\end{texcodeonly}

\begin{xltabular}{\textwidth}{YY}
  \caption{长表格} \\

    \toprule
    \textbf{Symptom} & \textbf{Metric} \\
    \midrule
  \endfirsthead % 以上设置第一张表表头

    \multicolumn{2}{c}{\tablename\ \thetable{}(续)} \\
    \toprule
    \textbf{Symptom} & \textbf{Metric} \\
    \midrule
  \endhead % 以上设置所有续表表头

    \bottomrule
    \multicolumn{2}{r}{表格待续} \\
  \endfoot % 以上设置第一张表结尾

    \bottomrule
  \endlastfoot % 以上设置最后一张表结尾

  % 开始表格内容
  Class that has many accessor methods and accesses a lot of external data & ATFD is more than a few\\
  Class that is large and complex & WMC is high\\
  Class that has a lot of methods that only operate on a proper & \\
  Class that has a lot of methods that only operate on a proper & \\
  Class that has a lot of methods that only operate on a proper & \\
  Class that has a lot of methods that only operate on a proper & \\
  Class that has a lot of methods that only operate on a proper & \\
  Class that has a lot of methods that only operate on a proper & \\
  Class that has a lot of methods that only operate on a proper & \\
  Class that has a lot of methods that only operate on a proper & \\
  Class that has a lot of methods that only operate on a proper & \\
  Class that has a lot of methods that only operate on a proper & \\
  Class that has a lot of methods that only operate on a proper & \\
  Class that has a lot of methods that only operate on a proper & \\
  Class that has a lot of methods that only operate on a proper & \\
  Class that has a lot of methods that only operate on a proper & \\
  Class that has a lot of methods that only operate on a proper & \\
  Class that has a lot of methods that only operate on a proper & \\
  Class that has a lot of methods that only operate on a proper & \\
  Class that has a lot of methods that only operate on a proper & \\
  Class that has a lot of methods that only operate on a proper & \\
\end{xltabular}
